\documentclass{article}
\usepackage[utf8x]{inputenc} 
\usepackage[english]{babel}
\usepackage{natbib}
\usepackage{booktabs}
\usepackage{setspace}\doublespacing
\setlength{\parskip}{0.5em}
\usepackage[left]{lineno}\linenumbers
\usepackage[a4paper, total={6in, 8in}]{geometry}
\usepackage{isomath}
\usepackage{amsmath}
\usepackage{mathtools}
\usepackage{longtable}
\usepackage[font={footnotesize,it}]{caption}
\usepackage{floatrow}
 
\usepackage[colorinlistoftodos]{todonotes}
\floatsetup[table]{capposition=top}


\begin{document}

\begin{center}
\large
\textbf{Adaptive variance in labile traits and its expected evolutionary change }
\end{center}


\bigskip

\noindent \textsuperscript{\textbf{1}} Centre for Biodiversity Dynamics (CBD), Department of Biology, Norwegian University of Science and Technology (NTNU), N-7491 Trondheim, Norway.

\noindent \textsuperscript{\textbf{2}} Corresponding author, email address: yimencr@gmail.com


\bigskip
\noindent \textbf{Running title}:  Evolution of variance components


\bigskip
\noindent \textbf{Keywords}: Variance, animal personality, price equation, quantitative genetics


\newpage
\section{Abstract}
Closer integration between behavioral ecology and quantitative genetics has a resulted in a recent increase in studies quantifying among- and within-individual sources of variation in labile traits. Consistent individual differences are commonly documented, and their existence is generally explained using adaptive arguments. However, the lack of a quantitative null model predicting the expected non-adaptive behavioral variation around an optimal phenotypic value makes it difficult to assess the adaptive nature of the observed patterns. We argue that estimating expected evolutionary change in trait variances across generations provides a way forward to test adaptive theory concerning behavioral variation. We describe an extension of the Price equation and the multivariate breeder’s equation that allows estimating the joint expected evolutionary change in the trait' mean, among- and within-individual variance. This framework provides a multi-level approach to the study the evolutionary ecology of labile traits that can be used to predict the expected changes in repeatability of behaviors and their coefficient of variation (CV). 


\newpage
\section{Introduction}
Evolutionary ecologists have traditionally been concerned with the evolution of the optimal mean phenotype in a population under a given a set of ecological conditions \citep{Krebs1997a}. Understanding the adaptive evolution of trait variances in response to ecological pressures has received less theoretical and empirical attention  \citep{Bulmer1971, Bull1987}. Recent calls for closer integration between behavioral ecology and quantitative genetics has resulted in an increased focus on the evolutionary ecology of behavioral variation within populations \citep{Westneat2010}, especially in the form of ‘animal personality’ \citep{Dingemanse2010, Dingemanse2013a}. A crucial but under examined claim arising from these studies is that consistent individual differences in behavior are adaptive (Bergmüller and Taborsky 2010; Dall et al. 2012; Dingemanse and Réale 2013), implying that among- and within-individual behavioral variance has been shaped by selection in a similar way as the mean behavior. This implies that adaptation will result in changes in a trait's repeatability and its heritability. However, few studies have estimated the expected evolutionary change in behavioral variance in response to selective pressures. 

The mere existence of non-zero among-individual differences in behavior or any other labile trait does not necessarily imply that this phenotypic variation is adaptive. The infinitesimal model from quantitative genetics provides the basis for a null expectation for the variance in a trait. First introduced by \cite{Fisher1918}, the infinitesimal model has proven extremely useful for estimating the additive genetic variance of phenotypes and predict evolutionary change in a wide array of study systems, particularly in breeding programs \citep{Barton2017}. It states that additive genetic variance in a quantitative trait can be quantified assuming that its expression is determined by an infinitely large number of genes and environmental factors each with very small effects. Such a combination of genetic and (short- and long-term) environmental effects could explain most observed individual differences in behavior without invoking any adaptive explanation. However, behavioral variance likely results from a combination of non-adaptive and adaptive process operating at different time scales. 

Variation in quantitative traits is ubiquitous. In quantitative genetic analyses of labile traits, variation among individuals is generally partitioned in its additive genetic, and permanent environmental components, while within individual variation is assumed to be underpinned by transient environmental effects and measurement error \citep{Lynch1998}. What maintains genetic variation has been a topic of discussion for a long time. Various non-adaptive processes, such as immigration and mutation, are thought to increase or maintain the amount of additive genetic variation in traits experiencing directional or stabilizing selection (see Lande 1992; Armbruster and Schwaegerle 1996; Whitlock et al. 2002).There is not yet, however, an unequivocal conclusion as to how the typical levels of genetic variance are maintained (e.g. Bürger, 2000 ; Johnson and Barton,2005 ; Zhang and Hill, 2005 a ; Hill, 2010). Much less attention has been paid to factors accounting for the magnitude of the environmental components of phenotypic variation. Since the genotype for the magnitude of environmental variation can be regarded as a quantitative trait, it may be assumed to be determined by the actions of multiple genes. Much of the standard methodology of quantitative genetics can then be invoked. Assuming that the environmental components of the phenotype have a genetic basis thus implies that the repeatability and heritability of the trait can evolve through selection on different components of phenotypic expression.

Phenotpic variation among-individuals within the same population will partly reflect plastic responses to environmental variation during development (see West-Eberhard 2003). If the optimal phenotype differs depending on the developmental environment, adaptive plasticity allows individuals to adjust their phenotype to match the optimum value for that environment, given the costs of plasticity and and environmental predictability \citep{Botero2014} Tufto 2015). Adaptive among-indivdiual variance can thus be driven by selection on plasticity, allowing individuals to move closer to the optimum phenotype given their developmental environment. However, not all non-genetic variation in behavior involves the predictable effects of adaptive plasticity. Non-adaptive random phenotypic variation can be caused by developmental instability because of unpredictable stochastic environmental effects (Falconer and Mackay 1996). While the evolution of adaptive plasticity has been the focus of much theoretical and empirical research, how selection shapes developmental instability causing among-individual phenotypic variation has received far less attention. Changes in developmental variance can also be caused by the evolution of population sensitivity to the environment via individual abilities to buffer phenotypic expression from stochastic environmental noise. The level of among individual variation associated to environmental effects is thus a combination of adaptive environmental canalization and adaptive phenotypic plasticity (Waddington 1942). This can also be viewed as evolution shaping the slope of the reaction norm allowing individuals to adapt to their developmental environment and also selection shaping an individual's absolute deviation from this population reaction norm.
 
In a similar fashion as developmental plasticity, reversible phenotypic plasticity allows individuals to get as close as possible to the optimal phenotypic expression by adjusting to its “current” environment. However, there is substantial within individual variation in phenotypic expression that does not seem to be associated to adaptive reversible plasticity. We expect environmental canalization to minimize any within-individual deviations in phenotypic expressions of behavior from the optimal reaction norm of each individual (see Westneat et al. 2015). In the context of adaptive consistent individual differences in behavior, the optimum level of within-individual behavioral variance may also depend upon the relative amount of among-individual behavioral variance. For instance, individual repeatability can have important implications for reliable communication concerning individual attributes (see Bradbury and Vehrencamp 2011), leading to hypothesis about the signaling advantages of consistent individual differences in signaling within social groups (Dall et al. 2012). 

Unfortunately, in most natural populations we have no quantitative predictions of the level of among- or within individual variation we expect to see around a mean phenotypic value under stabilizing selection, making it difficult to determine how much of the observed behavioral variation has an adaptive origin. In this paper we suggest studying the adaptive evolution of variation, by explicitly quantifying the expected evolutionary changes of the different sources of variation underpinning labile traits. We describe extensions to the Price equation (Price 1970) and the multivariate breeder’s equation that can be used to study evolutionary changes in trait variance at the among- and within-individual levels. These equations allow predicting the expected joint evolutionary change of the mean, among and within individual variance of labile traits across generations when stochastic perturbations have moved the population away from the optimal mean and/or variance or when the population is evolving towards a new optimal phenotypic distribution. 

\subsection{Evolutionary Change in Trait Variances}

Evolutionary quantitative genetics has mostly focused on the evolutionary dynamics of the mean phenotype ($\bar{z}$ of a population:
\begin{equation} \label{eq:mean}
\bar{z}= \frac{1}{n} \sum{z}
\end{equation}   

where $z$ is a vector of phenotypic values of a population with $n$ individuals.

The Price equation \citep{Price1970} can be used in the context of a quantitative genetics framework to describe evolutionary changes in mean phenotype from one time step to the next :

\begin{equation} \label{eq:Price}
\Delta \bar{z} = \frac{1}{\bar{w}} Cov(w,z)+E(w∆z)
\end{equation}

where the first term on the right-hand side represents natural selection as the covariance between fitness ($w$) and the individual trait value ($z$), standardized by the mean fitness in the population ($\bar{w}$), while the second term represents transmission bias as a fitness-weighted average of between-generation (i.e. parent-offspring) deviations in trait values. Here, our focus is on `the variance in a phenotypic trait:

\begin{equation} \label{eq:variance}
\sigma^{2}_{z} = \frac{1}{n} \sum{(z-\bar{z})^2}
\end{equation}

We can thus express changes in phenotypic variance using the Price equation \citep{Lehtonen2017} as:

\begin{equation} \label{eq:Price2}
\Delta \sigma_{z} = \frac{1}{\bar{w}} Cov(w,(z-\bar{z})^2)+E(w∆(z-\bar{z})^2)
\end{equation}

where we consider $(z-\bar{z})^2$ as a trait of the individual reflecting an absolute deviation from the population mean. Changes in the mean value of this new trait will result in changes in the variance of trait ($z$) in the population, because the variance in a population is by definition the average squared deviation from the mean (eq. \ref{eq:variance}).

We can then partition phenotypic expression into its among- and within-individual sources of phenotypic variation in a labile trait that is repeatedly expressed as, 

\begin{equation} 
	z=i + r, \label{eq:decomp1} 
\end{equation}

here, the expression of trait $z$ can be described as underpinned by factors creating individual differences $i$ which include genetic and developmental enviromental effects \citep{Falconer1996}, and the factors causing transient environmental effects ($r$). We can now use equations \ref{eq:decomp1} to expand \ref{eq:Price2} into a multi-level expression of the Price equation predicting changes in phenotypic trait variance:

\begin{equation} \label{eq:Price3}
\Delta \sigma_{z} = \frac{1}{\bar{w}} [Cov(w,i^2 )+Cov(w,E[r^2])]+E(w∆(z-\bar{z})^2)
\end{equation}

where $Cov(w,i^2)$ reflects selection associated to an individual’s average phenotypic (squared) deviation from the population mean value, and $Cov(w,E[r^2])$ selection on the susceptibility to environmental variation causing (squared) deviations of the different phenotypic expressions within an individual from its own mean.

For more convenient empirical use, we can express the causal processes affecting the evolution of phenotypic variance using a multiple regression equation \citep{Lande1983}, as:

\begin{equation} \label{eq:Breeders1}
\frac{w}{\bar{w}} = c + \beta_1 i + \beta_2 i^2  + \beta_{3} E(r^2)	
\end{equation}

where $\beta_1$ is the linear selection gradient relating the phenotype ($z$) of an individual to its relative fitness ($\frac{w}{\bar{w}}$), describing the strength of directional selection on the mean phenotype. The second term reflects the strength of selection on the squared deviation of an individual’s average phenotypic expression from the population mean, where $\beta_2$ is the selection gradient associated to the effects on fitness of $i^2$. This second term is thus equivalent to a quadratic selection gradient, with trait values being mean centered on the population mean in each selection episode (e.g., year). When this second term is negative, it reflects stabilizing selection, which it is expected to decrease variance. When it is positive it reflects disruptive selection, which is expected to increase variance. The third term reflects how the phenotypic deviations from an individuals own mean influence its fitness. In a simular way as with the $\beta_{2}$ negative values are exected to decrease the within individual variance while positive values increase it.

The expected evolution of the among- and within-individual variance components is thus a function of the selection gradients $\beta_2$ and $\beta_3$, respectively. These coefficients thereby allow empirically linking individual fitness to the individual’s average deviation from the population mean ($i^2$) and the average deviation of each expression from that individual’s own mean ($\bar{r}^2$). While theoretical models often use gaussian fitness functions in their evolutionary predictions and patterns of non-linear selection have been the focus of many studies \citep{Kingsolver2001}, their expected effect on the different components of phenotypic variance within populations has generally been overlooked, especially in empirical studies.  

The expected change on the mean, among- and within-individual variance in the phenotypic trait z can thus be estimated using the multivariate breeders equation \citep{Lande1983}. For this we need the genetic variance covariance matrix $G$ of $z$, $i^2$ and $\bar{r}^2$. If we denote $a_z$, $a_{i^2}$ and $a_{E(r^2)}$ as the breeding values for $z$, $i^2$ and $\bar{r}^2$ respectively, then G can be described as:

\begin{equation} \label{eq:G1}
\begin{bmatrix}
	a_{z} \\
	a_{i^2} \\
	a_{r^2} 
\end{bmatrix}
\sim MVN(0,G)
\end{equation}

were 

\begin{equation} \label{eq:G2}
G=
\begin{bmatrix}
	Va_{z}        & Cov_{a_z, a_{i^2}}   & Cov_{a_z, a_{r^2}}\\
	Cov_{a_{i^2}, a_z} & Va_{i^2}        & Cov_{a_{i^2}, a_{r^2}} \\
	Cov_{a_{r^2}, a_z} & Cov_{a_{r^2}, a_{i^2}} & Va_{r^2} \\
\end{bmatrix}
\end{equation}

If $\beta$ is the vector of selection gradients, $\beta_1$, $\frac{\beta_2}{2}$, $\frac{\beta_3}{2}$, then the expected evolutionary change in the mean ($\Delta_{\bar{z}}$), among ($\Delta_{\sigma^2_{i}}$) and within individual variance ()$\Delta_{\sigma^2_{r}}$) is,

 \begin{equation} \label{eq:Deltas}
 \begin{bmatrix}
 \Delta_{\bar{z}} \\
  \Delta_{\sigma^2_{i}} \\
 \Delta_{\sigma^2_{r}}
 \end{bmatrix}
 = \begin{bmatrix}
 \beta_{1} \\
 \frac{\beta_2}{2} \\
 \frac{\beta_3}{2}
 \end{bmatrix}
 \begin{bmatrix}
 V_{z}        & Cov_{z, i^2}   & Cov_{z, r^2}\\
 Cov_{i^2, z} & V_{i^2}        & Cov_{i^2, r^2} \\
 Cov_{r^2, z} & Cov_{r^2, i^2} & V_{r^2} \\
 \end{bmatrix}
 =\beta G
 \end{equation}.
 
It is important to note here that the coefficientes linking the different phenotypic values and their squared deviations to fitness are not strictly selection gradients as they are not variance standardized. We argue here that is important to retain the scale in order to study the joint evolution of trait mean and variances. 
 
\section{Discussion}
\cite{Wright1931} introduced the concept of adaptive topography by demonstrating that evolution of gene frequencies in a constant environment will maximize the mean fitness of the individuals in the population. Natural selection will cause changes gene frequencies moving populations upward on the adaptive topography, resulting in an increase in the mean fitness of the population. The formulation of such an adaptive topography has provided a theoretical foundation for analyses of many problems in evolutionary biology (Gavrilets 2004; Svensson and Calsbeek 2012). Lande (1976, 1979, 1982) and Lande and Arnold (1983) provided important conceptual and statistical advancements by developing a quantitative genetic theory for selection of the mean phenotype to study the adaptive topography for normally distributed characters and estimate the expected change in the mean of multivariate phenotypes. Here we describe the extension of this framework to incorporate the variance of a phenotype as part of its adaptive topography. While many of the components of the framework we are describing have been introduced elsewhere \citep{Hill2004,Hill2010, Lande1979, Lande1983}, we here provide a synthesis aimed at researchers studying the adaptive evolution of variance components. More specifically for researchers interested in studying the processes that increase/maintain variance among individuals in a population and also decrease within individual variance of repeatedly expressed traits.

Phenotypic variation on phenotypes under stabilizing selection is expected to decrease population mean fitness (Lande and Shannon 1996). Therefore, selection should minimize any phenotypic variance, especially in traits that are closely linked to fitness. A key test of the adaptive nature of behavioural variation is thus to quantify when larger values of phenotypic variance will lead to higher mean fitness in a population. Focusing on the evolutionary changes in phenotypic variance can provide some insights on the shape of the adaptive topography of populations.There are many reasons why we expect that the predicted changes in variance will not match the observed changes in natural populations. However, this also a feature of evolutionary predictions for changes in the mean phenotype of the population. Changes in the variance in the environment will also result in changes in the phenotypic variance without any selection shaping the mechanisms underpinning phenotypic variation. However, it is important to highlight that the mismatches between the predicted changes in the different components of phenotipyc variation and the change we may observed in natural population will provide useful insights in the biological processes shaping phenotypic distributions in natural populations. 

The statistical approach we advocate to study the expected adaptive evolutionary change on the among- and/or within-individual phenotypic variance in a population, hinges upon accurate estimates of additive genetic variation and selection gradients of the squared deviations of an individual’s mean behavior from the population mean ($i^2$), and the average deviations of repeated behavioral expressions from the individual’s own mean behavior $(E[r^2 ])$. If not properly modelled, inaccurate empirical estimates of mean individual behavior will underestimate the among individual variance in these quantities with cascading consequences on the estimates of selection gradients and additive genetic variance in the squared deviations of the average phenotype of an individual form the population mean and the average squared deviation from the repeated expressions from the individual’s own mean. These problems have been documented when using within-subject centering techniques \citep{Westneat2020}, and also when studying non-linear selection gradients \citep{Dingemanse2021}. To overcome these problems, we advocate the use of multivariate mixed models and/or error-in-variables models, as they provide a flexible and robust method for estimating accurate estimates of additive genetic variance and selection coefficients \citep{Dingemanse2021}.

\noindent\textbf{Evolution of genetic and developmental variance}

\noindent Following quantitative genetic theory, we can partition the among-individual sources of phenotypic variation in a labile trait that is repeatedly expressed as, 
\begin{subequations} 
	\begin{gather}
	z= i + r, \label{eq:decomp1a} \\
	i = a + d, \label{eq:decomp1b}
	\end{gather}
\end{subequations}

here, the expression of trait $z$ can be described as underpinned by factors creating individual differences $i$, and the factors causing transient environmental effects ($r$). The factors causing individual differences can in turn be partitioned into additive genetic ($a$), developmental enviromental ($d$) environmental effects (Falconer and Mackay 1996). The expected change in the direct genetic variance affecting a trait can be estimated based on the estimates of (non-linear) quadratic selection and additive genetic variance in the trait. The variance of a vector of squared values centered around zero equals two times the squared variance of the values, therefore the additive genetic variance in the squared deviation of an individual’s average trait from the population mean associated to direct genetic effects equals 2 times the squared additive genetic variance of the trait. Multiplying the quadratic selection gradient by two times the square of the additive genetic variances provides the expected change in additive genetic variance from one generation to the next. Assuming that there is only selection on the average squared deviation of an individual from the population mean (i.e $\beta_1=\beta_3=0$), there is no genetic variation in the squared deviations associated to developmental plasticity or the ability of individuals to buffer stochastic effects of the developmental environment in their phenotype, the expected change in the additive genetic variance could be described as

\begin{equation}
\Delta Va_z = \beta_2 Va_{z}^2
\end{equation}

In the absence of immigration and mutation, sustained directional or stabilizing selection for an optimum phenotypic value is intuitively expected to erode additive genetic variation. However, under the infinitesimal model, reductions in genetic variance associated with selection are expected to be mostly temporary (Bulmer 1981). This is because the extent of any permanent reduction in additive genetic variance due to selection decreases as the number of loci involved increases. In traits influenced by many loci, permanent reductions will therefore be small compared to temporary ones. Most of the reduction in genetic variance is expected to be temporary because it is caused by selection inducing correlations between pairs of loci leading to linkage disequilibrium. When selection ceases, the joint equilibrium between pairs of loci will be reestablished and the additive genetic variance will return to its baseline value. Therefore, even under strong selection, we always expect a certain amount of genetic variance in populations and thus some among individual variation if the average phenotype in the population is close to or fluctuates around the optimal phenotypic value.

Although the two environmental components of the phenotype (d and r) are not themselves heritable, the susceptibility of an individual to such environmental effects is a property that may have a genetic basis \citep{Bull1987}. For instance, when there is additive genetic variation in the sensitivity to the environment. We can describe phenotypic expression as 

\begin{equation} \label{eq:decomp2}
z=a + b_d x_d + b_r x_r
\end{equation}

where $d$ in equation \ref{eq:decomp1b} is now $b_d x_d$ in which $b_d$ is a vector of individual sensitivities to the developmental environment $x_d$. While $r$ in \ref{eq:decomp1a} is substituted for $b_r x_r$ in which $b_r$ is a vector of individual sensitivities to the current environment $x_r$. 

\begin{subequations} 
	\begin{gather}
	i^2 = (a + d)^2=(a + b_d x_d)^2, \label{eq:decomp3a} \\
	r^2 = (b_rx_r)^2 ,               \label{eq:decomp3b}
	\end{gather}
\end{subequations}

Estimating the additive genetic variance on an individual squared deviation from the population mean becomes more complicated. Animal breeders have attempted to estimate additive genetic variation in the environmental components on both phenotypes that are expressed only once and phenotypes that are repeatedly expressed \citep{Hill2010}. We expressed the environmental component in a reaction norm form. When the environmental variable is known and $b_d$ can be measured we can estimated the expected change in developmental variance, focusing on the expected evolutionary change in $b_d$, however most of variation in $d$ is likely to be the result of a probably large number of unknown environmental variables. The estimation of the additive genetic variation of the squared deviation of individual average expression from the popultion mean and the average squred deviation from its own mean is probably challenging specially, when there is additive genetic variation in the trait and there is also additive genetic variation in the environmental components of the trait.

SOMETHING ABOUT ESTIMATING INDIVIDUAL WITHIN INDIVIDUAL VARIANCE.

\bigskip
\noindent\textbf{Long term evolution}

\noindent Certain hypotheses concerning the evolution of phenotypic variance cannot be addressed by focusing on changes during single episodes of selection or from one generation to the next, because they operate over much longer evolutionary timescales. For instance, diversification bet-hedging favors genotypes that produce an array of different phenotypes at some short-term cost per generation of producing suboptimal phenotypes, because this reduces the long-term risk of a catastrophic loss of genotypic fitness from failing to have any phenotypes that match the environment in any one generation (see Simons 2011; Starrfelt and Kokko 2012). However, the fitness benefits of such long-term strategies may not always be apparent when focusing on changes in phenotypic variance from one generation to the next.

\noindent 
\section{Conclusion}
Understanding the forces that determine the magnitude of the non-genetic component of phenotypic variance is a broad question in evolutionary biology. Our understanding of why variances and heritabilities take the levels they do is at best, however, superficial. Formulating a null hypothesis for the expected levels of genetic or environmental variance in behavioral phenotypes within a population is often not possible. Therefore, exploring the selective pressures expected to produce adaptive short-term evolutionary changes in behavioral variation, both among and within individuals, is key to understanding the adaptive nature of consistent individual differences in behavior. We hope that the ideas outlined in this paper encourage behavioral ecologists to estimate the various selection gradients and the variance components necessary to quantify the expected evolutionary change in behavioral variation within populations. This will further our overall understanding of the ecological contexts expected to increase or decrease the different components of phenotypic variation in labile traits.

\bibliography{library}
\bibliographystyle{evolution}

\section{Supplementary appendix}
\begin{equation} \label{eq:decomp4}
i^2 = (a + d)^2= (a + b_d x_d)^2 = a^2 + 2(a + b_d x_d) + b_d x_d^2  
\end{equation}

The variance is then 

\begin{equation} \label{eq:variancesquared}
Va_i^2 = Va_z^2 + 4*(V_z + V_{bd}V_{xd} + Vxd) + (V_{bd}V_{xd} + Vxd)^2   
\end{equation}


\end{document}
